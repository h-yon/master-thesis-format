% !TEX root = ../main.tex
%\setlength{\textwidth}{15cm}
%\setlength{\textheight}{33\baselineskip}
%\setlength{\textwidth}{15cm}

\newcommand{\IIC}{I\raisebox{0.8ex}{\small 2}C}
%\newcommand{\figref}[1]{{\bf Figure~\ref{fig:#1}}~}
%\newcommand{\tabref}[1]{{\bf Table~\ref{#1}}~}
%\newcommand{\equref}[1]{{\bf Equation~\ref{#1}}~}
\newcommand{\figref}[1]{{図\ref{#1}}~}
\newcommand{\tabref}[1]{{表~\ref{#1}}~}
\newcommand{\equref}[1]{{式~\ref{#1}}~}
\newcommand{\chapref}[1]{第\ref{#1}章}
\newcommand{\secref}[1]{\ref{#1}節}
\newcommand{\sgn}{\mbox{sgn}}

%\newcommand{\eqref}[1]{式(\ref{eq:#1}) }
\newcommand{\bmath}[1]{\mbox{\boldmath $#1$}}

%%\pagestyle{fancyplain}
%
%\setlength{\baselineskip}{8mm}
%\setlength{\evensidemargin}{-3mm}
%\setlength{\oddsidemargin}{10mm}
%\setlength{\topmargin}{5mm}
%\setlength{\textheight}{33\baselineskip}
%\setlength{\textwidth}{15cm}
%
%\def\chapapp#1{第#1章 :}
%\renewcommand{\chaptermark}[1]{\markboth%
%{--- \chapapp{\thechapter}#1 ---}
%{--- \chapapp{\thechapter}#1 ---}}
%\renewcommand{\chaptermark}[1]{}
%\renewcommand{\sectionmark}[1]{}
%%\setlength{\headrulewidth}{1.0pt}
%\setlength{\headsep}{15mm}
%%\setlength{\footrulewidth}{0pt}
%%\setlength{\plainheadrulewidth}{0pt}
%%\setlength{\plainfootrulewidth}{0pt}
%%\addtolength{\headwidth}{\marginparsep}
%%\setlength{\headwidth}{\textwidth}
%%\lhead[\fancyplain{}{\bf\thepage}]{\fancyplain{}{\bf\leftmark}}
%%\chead{}
%%\rhead[\fancyplain{}{\bf\rightmark}]{\fancyplain{}{\bf\thepage}}
%%\lfoot{}
%%\cfoot{}
%%\rfoot{}
%
\newcommand{\pic}[4]{%
  \begin{figure}[htbp]%
    \begin{center}%
      \includegraphics[width=#2]{#1.#4}%
    \end{center}%
   \vspace{-5mm}
    \caption{#3}%
    \label{fig:#1}% 引用は\cite{hoge}
   \vspace{5mm}
  \end{figure}}

\newcommand{\pdf}[3]{%
  \pic{#1}{#2}{#3}{pdf}
}

\newcommand{\jpg}[3]{%
  \pic{#1}{#2}{#3}{jpg}
}

% \newcommand{\fig}[3]{%
%   \pic{#1}{#2}{#3}{eps}
% }

\newcommand{\png}[3]{%
  \pic{#1}{#2}{#3}{png}
}

% \newcommand{\figs}[5]{%
%   \begin{figure}[htbp]%
%     \leavevmode
%     \begin{center}%
%       \includegraphics[width=#2]{figure/#1.eps}%
%       \includegraphics[width=#4]{figure/#3.eps}%
%     \end{center}%
%     \vspace{-5mm}
%     \caption{#5}%
%     \label{fig:#1}% 本文で引用するときは\cite{fig:hoge}なので注意
%     \vspace{5mm}
%   \end{figure}}

% \newcommand{\twofigs}[6]{%
%   \begin{figure}[htbp]%
%     \begin{minipage}{0.5\hsize}%
%       \begin{center}%
%         \includegraphics[width=#2]{figure/#1.eps}%
%       \end{center}%
%       \caption{#3}%
%       \label{fig:#1}% 本文で引用するときは\cite{fig:hoge}なので注意
%     \end{minipage}%
%     \begin{minipage}{0.5\hsize}
%       \begin{center}%
%         \includegraphics[width=#5]{figure/#4.eps}
%       \end{center}%
%       \caption{#6}%
%       \label{fig:#4}% 本文で引用するときは\cite{fig:hoge}なので注意
%     \end{minipage}%
%   \end{figure}}

\newcommand{\twofigure}[3]{%
\begin{figure}\centering\begin{minipage}[b]{.5\linewidth}\centering #1\end{minipage}%
\begin{minipage}[b]{.5\linewidth}\centering #2\end{minipage}#3\end{figure}}
\newcommand{\twofigurec}[3]{%
\begin{figure*}\centering\begin{minipage}[b]{.5\linewidth}\centering #1\end{minipage}%
\begin{minipage}[b]{.5\linewidth}\centering #2\end{minipage}#3\end{figure*}}
\newcommand{\twofiguret}[3]{%
\begin{figure}\centering\begin{minipage}[b]{\linewidth}\centering #1\end{minipage}\\%
\begin{minipage}[b]{\linewidth}\centering #2\end{minipage}#3\end{figure}}

\RequirePackage{color}\definecolor{RED}{rgb}{1,0,0}\definecolor{BLUE}{rgb}{0,0,1}
%\providecommand{\DIFadd}[1]{{\protect\color{blue} \protect\ul{#1}}}
%\providecommand{\DIFadd}[1]{{\protect\color{blue} \fontsize{10.5pt}{0pt}#1}}
%\providecommand{\DIFdel}[1]{{\protect\color{red} \footnotesize \protect\Sl{#1}}}
%\providecommand{\DIFdel}[1]{{\protect\color{red} \footnotesize #1}}
\providecommand{\DIFaddbegin}{}
\providecommand{\DIFaddend}{}
\providecommand{\DIFdelbegin}{}
\providecommand{\DIFdelend}{}
\providecommand{\DIFaddFL}[1]{\DIFadd{#1}}
\providecommand{\DIFdelFL}[1]{\DIFdel{#1}}
\providecommand{\DIFaddbeginFL}{}
\providecommand{\DIFaddendFL}{}
\providecommand{\DIFdelbeginFL}{}
\providecommand{\DIFdelendFL}{}
